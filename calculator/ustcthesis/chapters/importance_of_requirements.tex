\chapter{需求分级}
\begin{table}[htbp]
\centering
\caption{需求分级表} \label{tab:classification}
\begin{tabular}{|c|c|c|}
    \hline
    需求类型 & 需求名称 & 需求分级 \\
    \hline
    基本运算 & 四则运算 & 必须的 \\
    \hline
    基本运算 & 乘方开方 & 必须的 \\
    \hline
    基本操作 & 复杂表达式输入 & 重要的 \\
    \hline
    基本运算 & 初等函数 & 重要的 \\
    \hline
    基本操作 & 清零 & 重要的 \\
    \hline
    基本操作 & BS & 重要的 \\
    \hline
    基本操作 & MS & 重要的 \\
    \hline
    基本操作 & M+ & 重要的 \\
    \hline
    基本操作 & M- & 重要的 \\
    \hline
    基本操作 & MS & 重要的 \\
    \hline
    基本操作 & ML & 重要的 \\
    \hline
    中级操作 & 绘制函数图像 & 重要的 \\
    \hline
    中级操作 & 统计工具 & 重要的 \\
    \hline
    中级操作 & 组合数学工具 & 重要的 \\
    \hline
    中级操作 & 进制转换 & 重要的 \\
    \hline
    中级操作 & 角度/弧度转换 & 必须的 \\
    \hline
    高级操作 & 极限 & 最好有的 \\
    \hline
    高级操作 & 微分 & 最好有的 \\
    \hline
    高级操作 & 积分 & 最好有的 \\
    \hline
    高级操作 & 常微分方程求解 & 最好有的 \\
    \hline
    高级操作 & 线性代数工具 & 最好有的 \\
    \hline
    高级操作 & FFT & 最好有的 \\
    \hline
    高级操作 & 卷积 & 最好有的 \\
    \hline
\end{tabular}
\end{table}

重要性分类如下:
\begin{itemize}
\item 必须的:绝对基本的特性,如果不包含,产品就会被取消。
\item 重要的:不是基本的特性:但这些特性会影响产品的生存能力。
\item 最好有的:期望的特性,但省略一个或多个这样的特性不会影响产品的生存能力
\end{itemize}
